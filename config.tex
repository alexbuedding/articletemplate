\documentclass[12pt]{article}                           % font size and type of document
\usepackage[a4paper, margin = 25mm]{geometry}           % paper size and margins
\usepackage[onehalfspacing]{setspace}                   % line spacing
\usepackage{lipsum}                                     % create dummy text in order to check                                                               % formating

\usepackage[T1]{fontenc}                                % hyphenating of words containing accented                                                          % chars, font encoding for accented chars
\usepackage{textcomp}                                   % additional symbols
\usepackage{mathptmx}                                   % font type (Times)

\usepackage[utf8]{inputenc}                             % used for mutated vowels e.g. ä, ü, ö, ß
\usepackage[ngerman]{babel}                             % load multilingual support for german
\usepackage[babel,german=guillemets]{csquotes}          % ensure multilingual support for biblatex                                                          % quoting

\newcommand{\titlename}{article template}               % change title of document
\newcommand{\authorname}{prename surname}               % and name of author here
                                                        % applied in whole document
\title{\titlename}
\author{\authorname}
\date{\today}
\newcounter{savepage}

\usepackage[hyphens]{url}                               % hyphening links
\usepackage{hyperref}
\hypersetup{
    colorlinks=true,
    linkcolor=black,
    citecolor=black,
    filecolor=magenta,      
    urlcolor=cyan,
    pdftitle={\titlename},
    pdfauthor={\authorname},
    pdfpagemode=FullScreen,
    }
\urlstyle{same}                                         % allows links within pdf document and to                                                                 external websites

\usepackage{xcolor}                                     % load xcolor package for customization
\usepackage{listings}                                   % supports listings environments
\definecolor{codegreen}{rgb}{0,0.6,0}                   % define and set custom colors
\definecolor{codegray}{rgb}{0.5,0.5,0.5}
\definecolor{codepurple}{rgb}{0.58,0,0.82}
\definecolor{backcolour}{rgb}{0.95,0.95,0.92}

\lstdefinestyle{mystyle}{                               % define listings style
    language={C++},
    morecomment=[l]{//},
    morekeywords={String},
    backgroundcolor=\color{backcolour},   
    commentstyle=\color{codegreen},
    keywordstyle=\color{magenta},
    numberstyle=\tiny\color{codegray},
    stringstyle=\color{codepurple},
    basicstyle=\ttfamily\footnotesize,
    breakatwhitespace=false,         
    breaklines=true,                 
    captionpos=b,                    
    keepspaces=true,                 
    numbers=left,                    
    numbersep=5pt,                  
    showspaces=false,                
    showstringspaces=false,
    showtabs=false,                  
    tabsize=2
}

\renewcommand{\lstlistlistingname}{Quellcodeverzeichnis}
\renewcommand{\lstlistingname}{Quellcode}               % change name of list of listings
\lstset{style=mystyle}                                  % set listings style to my defined style

\usepackage{graphicx}                                   % required for inserting images
\graphicspath{{images/}}
\usepackage{caption}                                    % more customization of captions
%\captionsetup{figurename=Abb., tablename=Tab.}
\usepackage{wrapfig}                                    % wrap text around figures
\DeclareCaptionType{mycapequ}[][Formelverzeichnis]      % adds list of equations
\DeclareCaptionLabelFormat{nan}{Gleichung #2}           % caption compatibility with equations 
\captionsetup[mycapequ]{labelformat=nan}

\usepackage[final]{pdfpages}                            % adds support for pdf file integration
\usepackage{tikz}                                       % create graphics in latex environment
\usetikzlibrary{positioning, calc}                      % enable relative positioning of nodes in tikz figure
\usepackage{pgfplots}                                   % create diagram, charts etc. with tikz library
\pgfplotsset{compat=1.18}                               % set used pgfplot package version for compatibility
\usepackage{amsmath}                                    % enhanced mathematical expressions
\usepackage{siunitx}                                    % includes si units support
\usepackage[ngerman, noabbrev]{cleveref}                % improves referencing of tables,                                                                      % figures, equations, etc.
%\crefformat{equation}{Gl.~(#2#1#3)}                    % customise reference to Gl. (default
                                                        %  "Gleichung")
\crefname{listing}{Quellcode}{Quellcodes}               % change reference name
\Crefname{listing}{Quellcode}{Quellcodes}

\usepackage{tocloft}                                    % extension for toc, lof, lot and other lists
\renewcommand{\cftsecleader}{\cftdotfill{\cftdotsep}}   % dot line for sections in toc
\usepackage{tocbibind}                                  % adds all lists to table of contents

\usepackage[acronym,toc,translate=babel,
            nonumberlist, nopostdot]{glossaries}
\renewcommand{\glsnamefont}[1]{\textbf{#1}}
\setlength\LTleft{0pt}
\setlength\LTright{0pt}
\setlength\glsdescwidth{0.8\hsize}                      % adds support for glossary and
\makenoidxglossaries                                         % acronyms and abbreviations
% add new acronyms here
\newacronym{utc}{UTC}{Coordinated Universal Time}
% add new glossary entries here
\newglossaryentry{latex}{name=LaTeX,description={Is a mark up language specially suited for scientific documents}}

\usepackage[                                            % header for bibliography
    backend=biber, 
    natbib=true,
    hyperref=true,
    style=ieee,
    sorting=none,
]{biblatex}
%\DeclareLanguageMapping{ngerman}{german-apa}           % only necessary for biblatex-apa style
%\DeclareFieldInputHandler{extradate}{\def\NewValue{}}  % remove suffix letters if author and year
\addbibresource{references.bib}                         % are identical in bibliography

\usepackage[activate={true,nocompatibility},            % activate protrusion and expansion
            final,                                      % enable microtype; use "draft" to disable
            tracking=true,                              % activate these techniques
            kerning=true,
            spacing=true,
            factor=1100,                                % add 10% protrusion amount (default is 
            stretch=10,                                 % 1000);reduce stretchability/shrinkability 
            shrink=10]{microtype}                       % (default is 20/20)
\SetProtrusion{encoding={*},family={bch},series={*},size={6,7}}
              {1={ ,750},2={ ,500},3={ ,500},4={ ,500},5={ ,500},
               6={ ,500},7={ ,600},8={ ,500},9={ ,500},0={ ,500}}
\SetTracking{encoding={*}, shape=sc}{40}
\SetExtraKerning[unit=space]
    {encoding={*}, family={qhv}, series={b}, size={large,Large}}
    {1={-200,-200}, \textendash={400,400}}              % microtype package improves general                                                            % appearence

\usepackage{afterpage}                                  % commands get expanded after the current                                                           % page  
\newcommand\blankpage{                                  % is output using \afterpage{\command}
    \newpage                                            % -> useful for \clearpage
    \null                                               % define command to insert blank page
    \thispagestyle{empty}
    \newpage}
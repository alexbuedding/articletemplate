% Universalmotor 20210503
\begin{tikzpicture}
        %\draw (-6,0) to [short,-](6,0);
        %\draw (0,6) to [short,-](0,-6);
        % Anker
        \draw[color = black] (4,0) node[elmech](motor){};
        \draw[color = black, thick](motor.top)
                (4,0.5) to [short,-](4,0.8){}
                (4,1.5)to [short,-](4,0.8)
        ;
        \draw[color = black, thick](motor.bottom)
                (4,-0.5) to [short,-](4,-0.8)
                to [short,-](4,-1.5)
        ;
        % Spannung am Motor
        \draw[thick,color = black]
                node[](A) at (4.7,0.5){}
                node[](B) at (4.7,-0.5){}
                (A) to [open, v^=\(\underline{U_\mathrm{i}}\)](B){}
        ;
        % Obere Grundlinie mit Strompfeil
        \draw[color = black, thick]
                (0,1.5) to [short,-,i_=\(\underline{I}\)] (0.7,1.5)
                to [L,l=\({R_\mathrm{A},}{L_\mathrm{A}}\)](2.3,1.5)
                to [short,-](4,1.5){}
                % Erregerwicklung
                (4,-1.5) to [short,-](2.3,-1.5)
                to [short,-](2.3,0){}
                (0.7,0)  to [L,l=\({R_\mathrm{f},}{L_\mathrm{f}}\)](2.3,0){}
                (0.7,0)  to [short,-](0.7,-1.5)
                to [short,-](0,-1.5){}
                % Spannungspfeil
                node[ocirc](C) at (0,1.5){}
                node[ocirc](D) at (0,-1.5){}
                (C) to [open, v = \(\underline{U}\)](D){}
        ;
\end{tikzpicture}
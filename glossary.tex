% add new glossary entries here
\newglossaryentry{latex}{name=LaTeX,description={LaTeX is a high-quality typesetting system; it includes features designed for the production of technical and scientific documentation. LaTeX is the de facto standard for the communication and publication of scientific documents. LaTeX is available as free software \cite{latex}.}}
\newglossaryentry{tikz}{name=TikZ,description={is a macro package for creating graphics. It is platform- and format-independent and works together with the most important TeX backend drivers, including pdfTeX and dvips. It comes with a user-friendly syntax layer called TikZ \cite{tikz}.}}
\newglossaryentry{pgfplots}{name=PGFPlots,description={draws high-quality function plots in normal or logarithmic scaling with a user-friendly interface directly in TEX. The user supplies axis labels, legend entries and the plot coordinates for one or more plots and pgfplots applies axis scaling, computes any logarithms and axis ticks and draws the plots. It supports line plots, scatter plots, piecewise constant plots, bar plots, area plots, mesh and surface plots, patch plots, contour plots, quiver plots, histogram plots, box plots, polar axes, ternary diagrams, smith charts and some more. It is based on Till Tantau's package pgf/TikZ \cite{pgfplots}}}
\newglossaryentry{circuitikz}{name=CircuiTikZ,description={The package provides a set of macros for naturally typesetting electrical and (somewhat less naturally, perhaps) electronic networks. It is designed as a tool that is easy to use, with a lean syntax, native to LaTeX, and directly supporting PDF output format. It has therefore been based on the very impressive PGF/TikZ package \cite{circuitikz}.}}
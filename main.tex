%---- general settings ----%

\documentclass[12pt]{article}                           % font size and type of document
\usepackage[a4paper, margin = 25mm]{geometry}           % paper size and margins
\usepackage[onehalfspacing]{setspace}                   % line spacing
\usepackage{lipsum}                                     % create dummy text in order to check                                                               % formating

\usepackage[T1]{fontenc}                                % hyphenating of words containing accented                                                          % chars, font encoding for accented chars
\usepackage{textcomp}                                   % additional symbols
\usepackage{mathptmx}                                   % font type (Times)

\usepackage[utf8]{inputenc}                             % used for mutated vowels e.g. ä, ü, ö, ß
\usepackage[ngerman]{babel}                             % load multilingual support for german
\usepackage[babel,german=guillemets]{csquotes}          % ensure multilingual support for biblatex                                                          % quoting

\newcommand{\titlename}{article template}               % change title of document
\newcommand{\authorname}{prename surname}               % and name of author here
                                                        % applied in whole document
\title{\titlename}
\author{\authorname}
\date{\today}
\newcounter{savepage}

\usepackage[hyphens]{url}                               % hyphening links
\usepackage{hyperref}
\hypersetup{
    colorlinks=true,
    linkcolor=black,
    citecolor=black,
    filecolor=magenta,      
    urlcolor=cyan,
    pdftitle={\titlename},
    pdfauthor={\authorname},
    pdfpagemode=FullScreen,
    }
\urlstyle{same}                                         % allows links within pdf document and to                                                                 external websites

\usepackage{xcolor}                                     % load xcolor package for customization
\usepackage{listings}                                   % supports listings environments
\definecolor{codegreen}{rgb}{0,0.6,0}                   % define and set custom colors
\definecolor{codegray}{rgb}{0.5,0.5,0.5}
\definecolor{codepurple}{rgb}{0.58,0,0.82}
\definecolor{backcolour}{rgb}{0.95,0.95,0.92}

\lstdefinestyle{mystyle}{                               % define listings style
    language={C++},
    morecomment=[l]{//},
    morekeywords={String},
    backgroundcolor=\color{backcolour},   
    commentstyle=\color{codegreen},
    keywordstyle=\color{magenta},
    numberstyle=\tiny\color{codegray},
    stringstyle=\color{codepurple},
    basicstyle=\ttfamily\footnotesize,
    breakatwhitespace=false,         
    breaklines=true,                 
    captionpos=b,                    
    keepspaces=true,                 
    numbers=left,                    
    numbersep=5pt,                  
    showspaces=false,                
    showstringspaces=false,
    showtabs=false,                  
    tabsize=2
}

\renewcommand{\lstlistlistingname}{Quellcodeverzeichnis}
\renewcommand{\lstlistingname}{Quellcode}               % change name of list of listings
\lstset{style=mystyle}                                  % set listings style to my defined style

\usepackage{graphicx}                                   % required for inserting images
\graphicspath{{images/}}
\usepackage{caption}                                    % more customization of captions
%\captionsetup{figurename=Abb., tablename=Tab.}
\usepackage{wrapfig}                                    % wrap text around figures
\DeclareCaptionType{mycapequ}[][Formelverzeichnis]      % adds list of equations
\DeclareCaptionLabelFormat{nan}{Gleichung #2}           % caption compatibility with equations 
\captionsetup[mycapequ]{labelformat=nan}

\usepackage[final]{pdfpages}                            % adds support for pdf file integration
\usepackage{tikz}                                       % create graphics in latex environment
\usetikzlibrary{positioning, calc}                      % enable relative positioning of nodes in tikz figure
\usepackage{pgfplots}                                   % create diagram, charts etc. with tikz library
\pgfplotsset{compat=1.18}                               % set used pgfplot package version for compatibility
\usepackage{amsmath}                                    % enhanced mathematical expressions
\usepackage{siunitx}                                    % includes si units support
\usepackage[ngerman, noabbrev]{cleveref}                % improves referencing of tables,                                                                      % figures, equations, etc.
%\crefformat{equation}{Gl.~(#2#1#3)}                    % customise reference to Gl. (default
                                                        %  "Gleichung")
\crefname{listing}{Quellcode}{Quellcodes}               % change reference name
\Crefname{listing}{Quellcode}{Quellcodes}

\usepackage{tocloft}                                    % extension for toc, lof, lot and other lists
\renewcommand{\cftsecleader}{\cftdotfill{\cftdotsep}}   % dot line for sections in toc
\usepackage{tocbibind}                                  % adds all lists to table of contents

\usepackage[acronym,toc,translate=babel,
            nonumberlist, nopostdot]{glossaries}
\renewcommand{\glsnamefont}[1]{\textbf{#1}}
\setlength\LTleft{0pt}
\setlength\LTright{0pt}
\setlength\glsdescwidth{0.8\hsize}                      % adds support for glossary and
\makenoidxglossaries                                         % acronyms and abbreviations
\input{acronyms}
% add new glossary entries here
\newglossaryentry{latex}{name=LaTeX,description={LaTeX is a high-quality typesetting system; it includes features designed for the production of technical and scientific documentation. LaTeX is the de facto standard for the communication and publication of scientific documents. LaTeX is available as free software \cite{latex}.}}
\newglossaryentry{tikz}{name=TikZ,description={is a macro package for creating graphics. It is platform- and format-independent and works together with the most important TeX backend drivers, including pdfTeX and dvips. It comes with a user-friendly syntax layer called TikZ \cite{tikz}.}}
\newglossaryentry{pgfplots}{name=PGFPlots,description={draws high-quality function plots in normal or logarithmic scaling with a user-friendly interface directly in TEX. The user supplies axis labels, legend entries and the plot coordinates for one or more plots and pgfplots applies axis scaling, computes any logarithms and axis ticks and draws the plots. It supports line plots, scatter plots, piecewise constant plots, bar plots, area plots, mesh and surface plots, patch plots, contour plots, quiver plots, histogram plots, box plots, polar axes, ternary diagrams, smith charts and some more. It is based on Till Tantau's package pgf/TikZ \cite{pgfplots}}}
\newglossaryentry{circuitikz}{name=CircuiTikZ,description={The package provides a set of macros for naturally typesetting electrical and (somewhat less naturally, perhaps) electronic networks. It is designed as a tool that is easy to use, with a lean syntax, native to LaTeX, and directly supporting PDF output format. It has therefore been based on the very impressive PGF/TikZ package \cite{circuitikz}.}}

\usepackage[                                            % header for bibliography
    backend=biber, 
    natbib=true,
    hyperref=true,
    style=ieee,
    sorting=none,
]{biblatex}
%\DeclareLanguageMapping{ngerman}{german-apa}           % only necessary for biblatex-apa style
%\DeclareFieldInputHandler{extradate}{\def\NewValue{}}  % remove suffix letters if author and year
\addbibresource{references.bib}                         % are identical in bibliography

\usepackage[activate={true,nocompatibility},            % activate protrusion and expansion
            final,                                      % enable microtype; use "draft" to disable
            tracking=true,                              % activate these techniques
            kerning=true,
            spacing=true,
            factor=1100,                                % add 10% protrusion amount (default is 
            stretch=10,                                 % 1000);reduce stretchability/shrinkability 
            shrink=10]{microtype}                       % (default is 20/20)
\SetProtrusion{encoding={*},family={bch},series={*},size={6,7}}
              {1={ ,750},2={ ,500},3={ ,500},4={ ,500},5={ ,500},
               6={ ,500},7={ ,600},8={ ,500},9={ ,500},0={ ,500}}
\SetTracking{encoding={*}, shape=sc}{40}
\SetExtraKerning[unit=space]
    {encoding={*}, family={qhv}, series={b}, size={large,Large}}
    {1={-200,-200}, \textendash={400,400}}              % microtype package improves general                                                            % appearence

\usepackage{afterpage}                                  % commands get expanded after the current                                                           % page  
\newcommand\blankpage{                                  % is output using \afterpage{\command}
    \newpage                                            % -> useful for \clearpage
    \null                                               % define command to insert blank page
    \thispagestyle{empty}
    \newpage}

%---- title page ----%

\begin{document}

%\title{Praxisphasenbericht Template}
%\author{Alex Buedding}
%\date{March 2023}
%\maketitle

\begin{titlepage}

\centering
\includegraphics[width=0.5\textwidth]{w-hs.png}
\par\vspace{5cm}
{\Huge\textbf{\titlename}\par}
\vspace{1cm}
{\large\scshape{\authorname}\par}
\vfill
{\large Datum: \today\par}

\end{titlepage}

%---- toc, lof, lot, loa, loe, lol ----%

\input{lists}

%---- text ----%

\setcounter{savepage}{\arabic{page}}
\pagenumbering{arabic}

\section{Equations}\label{sec:equations}

\begin{mycapequ}[!htbp]
    \begin{equation}
        {P(\bigcup_{n=1}^n A_n) \leq \sum_{n=1}^n P(A_n)}
        \label{eq:bool1} %\label used for referencing the equation in text
    \end{equation}
    \caption{Bool'sche Gleichung}
\end{mycapequ}

\begin{mycapequ}[!htbp]
    \begin{equation}
        \mathrm{E=m\cdot c^2}
        \label{eq:meequi} %\label used for referencing the equation in text
    \end{equation}
    \caption{Albert Einstein's mass-energy equivalence}
\end{mycapequ}

\clearpage

\section{Tables}\label{sec:tables}

\begin{table}[!htbp]
\centering
    \begin{tabular}{ | c | c | c | }
        \hline
        symbol & value & unit \\ \hline            
        $z Na$ & 11 & - \\ \hline      
        $z F$ & 9 & - \\ \hline      
        $Emax Na$ & 0.545 & $[MeV]$ \\ \hline
    \end{tabular}
    \caption{Beispieltabelle}
    \label{tab:example} %\label used for referencing the figure in text
\end{table}

\clearpage

\section{Graphs}\label{sec:graphs}

\subsection{TikZ}\label{subsec:TikZ}

\Gls{tikz}

\begin{figure}[!htbp]
    \centering
	\begin{tikzpicture}
        \draw (0,0) circle (1);
        \draw (2,0) circle (1.5in);
        \draw (5,0) ellipse (10pt and 20 pt);
        \draw node at (3,0) {$f(x)$};
        \filldraw (6,0) circle (0.1cm) node[anchor=west]{Anchored Node};
    \end{tikzpicture}
    \caption{\Gls{tikz} Beispielgrafik}
    \label{fig:tikzexamplegraphics}
\end{figure}

\begin{figure}[!htbp]
    \centering
    \begin{tikzpicture}[
                        youngnode/.style={rectangle, draw=red!60, fill=red!5, very thick, minimum size=40},
                        oldnode/.style={rectangle, draw=blue!60, fill=blue!5, very thick, minimum size=40},
                        ]
        %Nodes
        \node[oldnode]        (SusO)                            { $S_O(t)$};
        \node[oldnode]        (InfO)       [below=of SusO]      { $I_O(t)$};
        \node[oldnode]        (RecO)       [below=of InfO]      { $R_O(t)$};

        \node[youngnode]      (SusY)        [left=of SusO]      { $S_Y(t)$};
        \node[youngnode]      (InfY)        [left=of InfO]      { $I_Y(t)$};
        \node[youngnode]      (RecY)        [left=of RecO]      { $R_Y(t)$};

        %Lines
        \draw[->, very thick] (SusO.south east)  to node[right] {$a_{OO}$} (InfO.north east);
        \draw[->, very thick] (InfO.south)  to node[right] {$b_O$} (RecO.north);
        \draw[->, very thick] (RecO.east)  .. controls  +(right:17mm) and +(right:17mm)   .. (SusO.east);

        \draw[->, very thick] (SusY.south west)  to node[left] {$a_{YY}$} (InfY.north west);
        \draw[->, very thick] (InfY.south)  to node[left] {$b_Y$} (RecY.north);
        \draw[->, very thick] (RecY.west) .. controls  +(left:17mm) and +(left:17mm)   .. (SusY.west);

        \draw[dashed,->, very thick] (InfO.north west)  to  (SusY.south east);
        \draw[->, very thick] (SusY.south east)  to node[left] {$a_{OY}$} (InfY.north east);

        \draw[->, very thick] (SusO.south west)  to node[right] {$a_{YO}$} (InfO.north west);
        \draw[dashed,->, very thick] (InfY.north east)  to  (SusO.south west);
    \end{tikzpicture}
    \caption{\Gls{tikz} Beispieldiagramm}
    \label{fig:tikzexamplediagram}
\end{figure}

\clearpage

\subsection{PGFPlots}\label{subsec:pgfplots}

\Gls{pgfplots}

\begin{figure}[!htbp]
    \centering
    \begin{tikzpicture}
        \begin{axis}[clip=false,xmin=0,xmax=2.5*pi,ymin=-1.5,ymax=1.5, axis lines=middle,xtick={0,pi/2,pi,3*pi/2,2*pi},xticklabels={$0$,$\frac{\pi}{2}$,$\pi$,$\frac{3}{2}\pi$,$2\pi$},xticklabel style={anchor=south west},xmajorgrids=true,grid style=dashed]
            \addplot[domain=0:2*pi,red]{sin(deg(x))}
            node[right,pos=0.9]{$f(x)=\sin x$};
            \addplot[domain=0:2*pi,blue]{cos(deg(x))}
            node[right,pos=1.0]{$g(x)=\cos x$};
        \end{axis}
    \end{tikzpicture}
    \caption{\Gls{pgfplots} Beispiel einer 2D-Grafik}
    \label{fig:pgfplotsexample2D}
\end{figure}

\begin{figure}[!htbp]
    \centering
    \begin{tikzpicture}
        \begin{axis}[colormap/cool]
            \addplot3[mesh,samples=20]{1-x^2-y^2};
        \end{axis}
    \end{tikzpicture}
    \caption{\Gls{pgfplots} Beispiel einer 3D-Grafik}
    \label{fig:pgfplotsexample3D}
\end{figure}

\clearpage

\begin{figure}[!htbp]
    \centering
    \begin{tikzpicture}
        \begin{axis}[view={50}{30}]
            \addplot3+[domain=0:5*pi,samples=60,samples y=0]({sin(deg(x)},{cos(deg(x)},{x});
        \end{axis}
    \end{tikzpicture}
    \caption{Weiteres \Gls{pgfplots} Beispiel einer 3D-Grafik}
    \label{fig:anotherpgfplotsexample3D}
\end{figure}

\begin{figure}[!htbp]
    \centering
    \begin{minipage}[t]{0.95\textwidth}
        \centering
        \begin{tikzpicture}
            \begin{axis}[xmin=0, xmax=32, xlabel=$k$, ylabel=$x_k$, ymin=0, ymax=1, ymajorgrids=true, xmajorgrids=true, width=0.99\linewidth]
                \addplot+[only marks] table[x=k, y=hk]{data/exampledata.txt};
            \end{axis}
        \end{tikzpicture}
        \caption{\Gls{pgfplots} Beispiel einer Datenreihe}
        \label{fig:pgfplotsexampledata}
    \end{minipage}
\end{figure}

\clearpage

\subsection{CircuiTikZ}\label{subsec:circuitikz}

\Gls{circuitikz}

\begin{figure}[!htbp]
    % Universalmotor 20210503
\begin{tikzpicture}
        %\draw (-6,0) to [short,-](6,0);
        %\draw (0,6) to [short,-](0,-6);
        % Anker
        \draw[color = black] (4,0) node[elmech](motor){};
        \draw[color = black, thick](motor.top)
                (4,0.5) to [short,-](4,0.8){}
                (4,1.5)to [short,-](4,0.8)
        ;
        \draw[color = black, thick](motor.bottom)
                (4,-0.5) to [short,-](4,-0.8)
                to [short,-](4,-1.5)
        ;
        % Spannung am Motor
        \draw[thick,color = black]
                node[](A) at (4.7,0.5){}
                node[](B) at (4.7,-0.5){}
                (A) to [open, v^=\(\underline{U_\mathrm{i}}\)](B){}
        ;
        % Obere Grundlinie mit Strompfeil
        \draw[color = black, thick]
                (0,1.5) to [short,-,i_=\(\underline{I}\)] (0.7,1.5)
                to [L,l=\({R_\mathrm{A},}{L_\mathrm{A}}\)](2.3,1.5)
                to [short,-](4,1.5){}
                % Erregerwicklung
                (4,-1.5) to [short,-](2.3,-1.5)
                to [short,-](2.3,0){}
                (0.7,0)  to [L,l=\({R_\mathrm{f},}{L_\mathrm{f}}\)](2.3,0){}
                (0.7,0)  to [short,-](0.7,-1.5)
                to [short,-](0,-1.5){}
                % Spannungspfeil
                node[ocirc](C) at (0,1.5){}
                node[ocirc](D) at (0,-1.5){}
                (C) to [open, v = \(\underline{U}\)](D){}
        ;
\end{tikzpicture}
    \caption{Universalmotor}
    \label{fig:universalmotor}
\end{figure}

\begin{figure}[!htbp]
    % Spartransformator 20220405
\begin{tikzpicture}
    \draw[color = black, thick]
        % Spulen
        (0,3) to [L,v = \(\Delta U \),l =\(\Delta N \),-*](0,0)
        to [L,l =\(N_2\),-*](0,-3){}
        % Anschluss links
        (-3, 3) to [short,-,i_= \(I_1\)](0,3){}
        (-3,-3) to [short,-](0,-3)
        node[ocirc](A) at (-3, 3){}
        node[ocirc](B) at (-3,-3){}
        (A) to [open, v= \(U_{1}\)](B)
        % Anschluss rechts
        (0, 0) to [short,-,i_= \(I_2\)](2,0){}
        (0,-3) to [short,-](2,-3){}
        node[ocirc](C) at (2, 0){}
        node[ocirc](D) at (2,-3){}
        (C) to [open, v^= \(U_{2}\)](D)
        % Klammer
        (-1.1,2.2) to [short,-](-1.2,2.1)
        to [short,-](-1.2,0.1)
        to [short,-](-1.3,0)node[left]{\(N_1\)}
        to [short,-](-1.2,-0.1)
        to [short,-](-1.2,-2.1)
        to [short,-](-1.1,-2.2){}
    ;
 \end{tikzpicture}
    \caption{Spartransformator}
    \label{fig:autotransformer}
\end{figure}

\begin{figure}[!htbp]
    \begin{tikzpicture}
  \draw[color=black, thick] % transformer T-Equivalent circuit with R, L-üM and üM
    % Grundlinie
    (-6,0) to [short,-] (6,0){}
    % Laengszweig links
    (-6,3) to [short, i_=\(i_1\), -] (-5,3)
    (-5,3) to [R, l=\(R_1\)]  (-3.5,3)
    to [L,l=\(L_1 - \ddot{u}M\),](-0.4,3)
    to [short](0,3)
    % Laengszweig rechts
    to [short](0.4,3)
    to [L,l=\(L^*_2 - \ddot{u}M\),](3.5,3)
    to [R,l=\(R^*_2\)] (5,3)
    to [short, i_=\(i^*_2\),-] (6,3){}% node{}
    % Querzweig
    (0,3) to [L, l=\(\ddot{u}M\), i>^=\(i_1-i^*_2\), *-*](0,0)
    % Anschlüsse
    node[ocirc] (A) at (-6,3) {}
    node[ocirc] (B) at (-6,0) {}
    (A) to [open, v=\(u_1\)] (B)
    node[ocirc] (C) at (6,3) {}
    node[ocirc] (D) at (6,0) {}
    (C) to [open, v^=\(u^*_2\)] (D)
  ;
\end{tikzpicture}

    \caption{Transformator T-Ersatzschaltbild mit L, ü und M}
    \label{fig:transformertequivalent}
\end{figure}

\begin{figure}[!htbp]
    % Transformator T-Ersatzschaltbild mit R, Lsigma und L1h 20201120
\begin{tikzpicture}
       \draw[color=black, thick]
              % Grundlinie
              (-6,0) to [short,-] (6,0){}
              % Laengszweig links
              (-6,3) to [short, i_=\(i_1\), -] (-5,3)
              to [R, l=\(R_1\)]  (-3.5,3)
              to [L,l=\(L_{1 \sigma}\),](-0.4,3)
              to [short](0,3)
              % Laengszweig rechts
              to [short](0.4,3)
              to [L,l=\(L'_{2 \sigma}\),](3.5,3)
              to [R,l=\(R'_2\)] (5,3)
              to [short, i_=\(i'_2\),-] (6,3){}
              % Querzweig
              (0,3) to [L, l=\(L_{1\mathrm{h}}\), i>^=\(i_\mu\), *-*](0,0)
              % Anschlüsse
              node[ocirc] (A) at (-6,3) {}
              node[ocirc] (B) at (-6,0) {}
              (A) to[open, v=\(u_1\)] (B)
              node[ocirc] (C) at (6,3) {}
              node[ocirc] (D) at (6,0) {}
              (C) to[open, v^=\(u'_2\)] (D)
       ;
\end{tikzpicture}

    \caption{Transformator T-Ersatzschaltbild mit L und $\sigma$}
    \label{fig:transformertequivalent2}
\end{figure}

\begin{figure}[!htbp]
    % Transformator vereinfachtes T-Ersatzschaltbild 1 20201120
\begin{tikzpicture}
       \draw[color = black, thick]
              % Grundlinie
              (-3.9,0) to [short,-] (6,0){}
              % Laengszweig links
              (-3.9,3) to [short, i_= \(i_1\)] (-2.9,3)
              to [R, l= \(R_1\),] (-0.4,3)
              to [short] (0,3)
              % Laengszweig rechts
              to [short] (0.4,3)
              to [L, l=\(L_1 {\frac{\sigma}{1 - \sigma}}\)] (2.9,3)
              to [R, l = \( R'_2{(1 + \sigma_1)^{2}} \)] (4.6,3)
              to [short,-,i_= \(\frac{i'_2}{(1+\sigma_1)}\)] (6,3){}
              % Querzweig
              (0,3) to [L, l=\(L_1\), i>^=\(i_0\), *-*](0,0)
              % Anschlüsse
              node[ocirc](A) at (-3.9,3){}
              node[ocirc](B) at (-3.9,0){}
              (A) to [open, v=\(u_1\)](B)
              node[ocirc](C) at (6,3){}
              node[ocirc](D) at (6,0){}
              (C) to  [open, v^=\(u'_2(1 + \sigma_1)\)](D)
       ;
\end{tikzpicture}
    \caption{Vereinfachtes Transformator T-Ersatzschaltbild mit zusammengefasster Induktivität auf Sekundärseite}
    \label{fig:transformertequivalent3}
\end{figure}

\begin{figure}[!htbp]
    % Transformator vereinfachtes T-Ersatzschaltbild 2 20201120
\begin{tikzpicture}
   \draw[color = black, thick]
      % Grundlinie
      (-6,0) to [short,-] (3.9,0){}
      % Laengszweig links
      (-6,3) to [short, i_=\(i_1\), -] (-5,3)
      (-5,3) to [R, l=\(R_1\)]  (-3.5,3)
      to [L,l=\(\sigma L_1\),](-0.4,3)
      to [short](0,3)
      % Laengszweig rechts
      to [short] (0.4,3)
      to [R, l = \(R'_2 / (1 + \sigma_2)^{2}\)] (2.9,3)
      to [short,-,i_={\tiny \(i'_2  (1+\sigma_2)\)}] (3.9,3){}
      % Querzweig
      (0,3) to [L, l=\((1 - \sigma) L_1\), i>^=\(i_0\), *-*](0,0)
      % Anschlüsse
      node[ocirc](A) at (-6,3){}
      node[ocirc](B) at (-6,0){}
      (A) to [open, v=\(u_1\)](B)
      node[ocirc](C) at (3.9,3){}
      node[ocirc](D) at (3.9,0){}
      (C) to [open, v^=\(\dfrac{u'_2}{1 + \sigma_2}\)](D)
   ;
\end{tikzpicture}
    \caption{Transformator T-Ersatzschaltbild mit zusammengefasster Induktivität auf Primärseite}
    \label{fig:transformertequivalent4}
\end{figure}

\clearpage

% --- Bibliography --- %

\pagenumbering{Roman}
\setcounter{page}{\thesavepage}

\printbibliography[heading=bibintoc,title={Literaturverzeichnis}]

\clearpage

%---- glossary ----%

\printnoidxglossary[style=long,title={Glossar}]

\clearpage

\end{document}
